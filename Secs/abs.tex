% ------------------------------------------------------------------------
% ------------------------------------------------------------------------
% ------------------------------------------------------------------------
%                                Abstract
% ------------------------------------------------------------------------
% ------------------------------------------------------------------------
% ------------------------------------------------------------------------
\chapter*{ABSTRACT}

\footnotesize{
\begin{description}
  \item[TITLE:] THEORETICAL INVESTIGATION OF THE RRUDLESDEN-POPPER OXINITRIDES STRUCTURES\\ $Sr_{2}(Ta,Nb)O_{4-x}N_{x}$ VIA DENSITY FUNCTIONAL THEORY \astfootnote{B. Sc. Thesis}
  \item[AUTHOR:] JUAN ALEJANDRO PINTO CASTRO\asttfootnote{Faculty of Sciences. School of Physics. Advisor: Andrés Camilo García Castro, Ph.D in Physics.}
  \item[KEYWORDS:] PEROVKSITE, RRUDLESDEN-POPPER, DENSITY FUNCTIONAL THEORY, FERROELECTRICITY \& PHOTOCATALYSIS.
  \item[DESCRIPTION:]\hfill \\ Ruddlesden-Popper phase (\textsc{rp}) is a type of perovskite arranged in dislocated layers, considered as an intergrowth of $n$  layers of perovskite $ABX_{3}$ and a layer $AX$:$(AX)(ABX_{3})$.
  The partial replacement of oxygen, $O^{2-}$, by nitrogen, $N^{3-}$, in metal oxides (based on their similarities in electronegativity, polarizability, ionic radius, and coordination number) entails a modification of its physical properties associated with electrical, optical, catalytic and magnetic responses. Additionally, a possible ferroelectric behavior induced by this substitution is anticipated.
  The current work presents the theoretical analysis of the structural, electronic and phononic properties of the compound $Sr_{2}(Ta,Nb)O_{4-x}N_{x}$ for two anionic concentrations $x=0.5$ y $x=1.0$. The previous analysis was achieved through the use of density functional theory, \textsc{dft}, implemented in the Vienna ab initio simulation package (\textsc{vasp}) compiled in the \texttt{"Guane"} cluster of the University supercomputer Santander industrial (\textsc{uis}).
  
  In detail, in order to cover a complete assembly of possible anionic combinations, the anionic substitution was operated through the code \textsc{sod} obtaining 11 configurations with concentration $x=1.0$, and two configurations with concentration $x=0.5$, all with different anion order.
  Subsequently, the structural relaxation was carried out, finding two configurations of minimum energy with anionic ordering \emph{trans} and \emph{cis}, belonging to the space group $Immm$ (S.G. 71) and $Cmcm$ (S.G. 63), respectively.
  Calculations from \textsc{dft} indicate that the anionic order \emph{cis} is energetically more favorable than the anionic order \emph{trans}.
  The electronic structure suggests an insulating material with a reduction of the bandgap in the point $\Gamma$ of the Brillouin zone due to the introduction of nitrogen in the structure, allowing the use of light in the visible range for the division of water by photocatalysis.
  The phononic structure shows imaginary modes that lead to displacement of the transition metal, and therefore a drop in spatial symmetry, achieving a phase transition to the most favorable chiral smooth mode belonging to the space group $C222_ {1}$ (SG. 20). 
\end{description}}\normalsize
% ------------------------------------------------------------------------ 