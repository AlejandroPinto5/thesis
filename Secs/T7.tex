% ------------------------------------------------------------------------
% ------------------------------------------------------------------------
% ------------------------------------------------------------------------
%                             Conclusiones
% ------------------------------------------------------------------------
% ------------------------------------------------------------------------
% ------------------------------------------------------------------------

\chapter{RECOMENDACIONES Y TRABAJO FUTURO}
% ------------------------------------------------------------------------

Como se pudo apreciar en la estructura electrónica de la configuración con ordenamiento \emph{trans-}$Sr_{2}(Ta,Nb)O_{3}N$ y grupo espacial $Immm$ (S.G. 71), los resultados de la brecha entre la banda de conducción y la banda de valencia en el punto de alta simetría $\Gamma$ no son acordes a los reportes en la literatura \cite{Cen2019OptimizedSplitting,Clarke2002,Bouri2018}, obteniendo una brecha mucho menor a lo esperado. Esto sugiere un aumento del parámetro \textsc{u} mayor a $4eV$ con el fin de subir en energía las bandas del metal de transición, y así lograr resultados mas acordes a la literatura.

Por otro lado, se encontró que la fase $Sr_{2}(Ta,Nb)O_{3}N$ con ordenamiento aniónico \emph{cis}, desplazamiento propio $\Gamma_{1}^{-}$ y grupo espacial $C222_{1}$ (S.G. 20), no hace parte del grupo de punto polar, indicando que no es un material ferroeléctrico. Sin embargo, se sugiere realizar cálculos de primeros principios para analizar si el material puede tener un doble pozo, además de sugerir la medición de la polarización espontánea. Un resultado interesante es la mención de que el material pertenece al grupo de punto enantomórfo, lo que sugiere sea un material quiral. Los materiales quirales se distinguen por no poder superponerse a su imagen espejo
\cite{milic2021Multi}, es decir, sus operaciones de simetría no contienen ejes de rotación impropios. Estos materiales son muy deseados por sus aplicaciones en spintrónica y quiroptoelectrónica debido al fuerte acoplamiento espín-orbita y controlable dispersión Rashba \cite{milic2021Multi}.






\chapter{CONCLUSIONES}

En este trabajo de grado se investigaron de manera computacional y en el marco de la materia condensada y la teoría funcional de la densidad las propiedades estructurales, electrónicas y fonónicas del material perovskita tipo Ruddlesden-Popper $Sr_{2}(Ta,Nb)O_{3}N$, además de reportar la necesidad de introducir la teoría del campo medio dinámico (por sus siglas es inglés \textsc{dmft}) a la estructura $Sr_{2}(Ta,Nb)O_{3.5}N_{0.5}$.

Se encontraron once configuraciones con diferente ordenamiento aniónico para el material $Sr_{2}(Ta,Nb)O_{3}N$, y dos configuraciones para el material $Sr_{2}(Ta,Nb)O_{3.5}N_{0.5}$.  Se comprendió la discrepancia sobre la energía estructural mas favorable entre los ordenamientos aniónicos \emph{trans} y \emph{cis}, concluyendo que el problema radicaba en una cuestión de simetría para que el orden \emph{cis} se diera en estos materiales, siendo este el ordenamiento mas favorable con energia de $-55.41eV$ y $-52.16eV$ para $Sr_{2}TaO_{3}N$ y $Sr_{2}NbO_{3}N$, respectivamente. Se pudo apreciar el papel que juega el parámetro \textsc{u}$=4eV$ en elementos altamente correlacionados como el tántalo y el niobio, obteniendo energías estructurales y bandas electrónicas mas acordes a lo reportado en la literatura.

En la estructura electrónica de las configuraciones $Immm(71)$ y $Cmcm(63)$ se encontró el posicionamiento del metal de transición en las bandas de conducción, mientras que las bandas de valencia son ocupadas por el nitrógeno, disminuyendo la  brecha directa en el punto $\Gamma$ de alta simetría en la zona de Brillouin, lo que hace a este material favorable para la división de agua mediante fotocatálisis.

Al realizar la curva de dispersión de fonones, se encontraron mas modos inestables en el orden \emph{cis} comparado con el orden \emph{trans}, además de hallar el modo mas estable después de condensar los modos imaginarios, concluyendo que el desplazamiento propio es $\Gamma_{1}^{-}$ con grupo espacial $C222_{1}$ (S.G. 20). Aunque este grupo no pertenece a los grupos de punto polar, hace parte del grupo de punto quiral, abriendo la puerta al análisis de sus características estructurales.

Cabe resaltar que los resultados de esta investigación fueron presentados como poster en el congreso internacional \textsc{virtual mrs spring meeting $\&$ exhibit} en abril del 2021, con el título \textit{EL09.07.05: THEORETICAL INVESTIGATION OF THE ELECTRONIC PROPERTIES OF RUDDLESDEN-POPPER Sr$_{2}$AO$_{4-x}$N$_{x}$(A=Nb,Ta)(x=1.0)}.






% ------------------------------------------------------------------------
% ------------------------------------------------------------------------
