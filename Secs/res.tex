% ------------------------------------------------------------------------
% ------------------------------------------------------------------------
% ------------------------------------------------------------------------
%                                Resumen
% ------------------------------------------------------------------------
% ------------------------------------------------------------------------
% ------------------------------------------------------------------------
\chapter*{RESUMEN}

\footnotesize{
\begin{description}
  \item[TÍTULO:] INVESTIGACIÓN TEÓRICA DE LAS ESTRUCTURAS RUDDLESDEN-POPPER DE LOS OXINITRUROS $Sr_{2}(Ta,Nb)O_{4-x}N_{x}$($x=0.5$ Y $1.0$) A TRAVÉS DE LA TEORÍA FUNCIONAL DE LA DENSIDAD\astfootnote{Trabajo de grado}
  \item[AUTOR:] JUAN ALEJANDRO PINTO CASTRO\asttfootnote{Facultad de Ciencias Básicas. Escuela de Física, Física computacional en Materia Condensada. Director: Andrés Camilo García Castro, Doctorado en Física.}
  \item[PALABRAS CLAVE:] PEROVSKITA, RUDDLESDEN-POPPER, TEORÍA FUNCIONAL DE LA DENSIDAD, FERROELECTRICIDAD, FOTOCATÁLISIS.
  \item[DESCRIPCIÓN:]\hfill \\ La fase Ruddlesden-Popper (\textsc{rp}) es un tipo de perovskita dispuesta en capas dislocadas, considerada como un intercrecimiento de $n$ capas de perovskita $ABX_{3}$ y una capa $AX$: $(AX)(ABX_{3})$. 
  La sustitución parcial de oxígeno, O$^{2-}$, por nitrógeno, N$^{3-}$, en óxidos metálicos (basados en sus similitudes en electronegatividad, polarizabilidad, radio iónico y número de coordinación) conlleva a  modificaciones de sus propiedades físicas asociadas a respuestas eléctricas, ópticas, catalíticas y magnéticas. Adicioalmente, se prevee un posible comportamiento ferroeléctrico inducido por esta sustitución. 
  En el actual trabajo se presenta el análisis teorico de las propiedades estructurales, electrónicas y fonónicas del compuesto $Sr_{2}(Ta,Nb)O_{4-x}N_{x}$ para dos concentraciones aniónicas $x=0.5$ y $x=1.0$. El anteior analisis logrado mediante el uso de teoría funcional de la densidad, DFT, implementada en el paquete de simulación Vienna ab initio (\textsc{vasp}) compilado en el cluster \texttt{"Guane"} de la supercomputadora de la Universidad industrial de Santander (\textsc{uis}).
  
  En detalle, con el fin de cubrir un completo ensamble de combinaciones anionicas posibles, se realizó la sustitución aniónica a traves del código \textsc{sod} obteniendo 11 configuraciones con concentración $x=1$, y dos configuraciones con concentración $x=0.5$, todas con diferente orden aniónico. 
  Posteriormente se realizó la relajación estructural, encontrando dos configuraciones de mínima energía con ordenamiento aniónico \emph{trans} y \emph{cis}, pertenecientes al grupo espacial $Immm(71)$ y $Cmcm(63)$, respectivamente. 
  Cálculos de \textsc{dft} indican que el orden aniónico \emph{cis} es energéticamente más favorable que el orden aniónico \emph{trans}. 
  La estructura electrónica sugiere un material aislante con una reducción del bandgap en el punto $\Gamma$ de la zona de Brillouin debido a la introducción del nitrógeno en la estructura, permitiendo el uso de luz en el rango visible para la división de agua mediante fotocatálisis. 
  La estructura fonónica evidencia modos imaginarios que llevan a desplazamientos del metal de transición, y por ende una caída de la simetría espacial, logrando una transición de fase al más favorable modo suave quiral perteneciente al grupo espacial $C222_{1}$ (SG. 20).

\end{description}}\normalsize
% ------------------------------------------------------------------------ 