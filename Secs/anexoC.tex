% ------------------------------------------------------------------------
% ------------------------------------------------------------------------
% ------------------------------------------------------------------------
%                                Anexo C
% ------------------------------------------------------------------------
% ------------------------------------------------------------------------
% ------------------------------------------------------------------------
% ------------------------------------------------------------------------
\newpage
\anexo{Grupos de punto polar y enantomórfico}\label{anexoC}
% ------------------------------------------------------------------------
Uno de los mas importantes conceptos en el campo de ciencia de materiales es la simetría, que juega un papel importante en la construcción de propiedades estructurales. En cristalográfica, hay 32 grupos de punto y 230 grupos espaciales que describen las estructuras cristalinas. En especial, la simetría de inversión espacial, la cual indica inmutabilidad bajo inversión de coordenadas espaciales, es de suma importancia en la clasificación de materiales férricos . Entre los 32 grupos de punto , 11 clases centro-simétricas tienen simetría de inversión espacial, mientras que los 21 restantes son no-centro-simétricos \cite{Shi2016SymmetryFerroelectrics}. Los grupos de punto polar que pueden contener ferroelectricidad son: 

\begin{table}[H]
\centering
\begin{tabular}{lrc}
\hline
\multicolumn{1}{c}{\multirow{2}{*}{Sistema cristalino}} & \multicolumn{2}{c}{\multirow{2}{*}{Grupo de punto enantiomórfo}} \\
\multicolumn{1}{c}{}                                    & \multicolumn{2}{c}{}                                             \\ \hline
Triclínico                                              & 1                                       &                        \\
Monoclínico                                             & 2                                       & m                      \\
Ortorómbico                                             &                                         & mm2                    \\
Tetragonal                                              & 4                                       & 4mm                    \\
Trigonal                                                & 3                                       & 3m                     \\
Hexagonal                                               & 6                                       & 6mm                    \\
Cúbico                                                  & \multicolumn{1}{c}{}                    &                        \\ \hline
\end{tabular}
\caption{Grupos de punto polar en cada sistema cristalino en la notación de Hermann-Mauguin \cite{Hermann2015notation}.}
\label{Tab. polar}
\end{table}

Los grupos de puntos que no poseen rotaciones impropias se denominan enantomórficos. Los grupos de punto enantomórfico que pueden contener quiralidad son: 

\begin{table}[H]
\centering
\begin{tabular}{lrc}
\hline
\multicolumn{1}{c}{\multirow{2}{*}{Sistema cristalino}} & \multicolumn{2}{c}{\multirow{2}{*}{Grupo de punto enantiomórfo}} \\
\multicolumn{1}{c}{}                                    & \multicolumn{2}{c}{}                                             \\ \hline
Triclínico                                              & 1                              &                                 \\
Monoclínico                                             & 2                              &                                 \\
Ortorómbico                                             &                                & 222                             \\
Tetragonal                                              & 4                              & 422                             \\
Trigonal                                                & 3                              & 32                              \\
Hexagonal                                               & 6                              & 622                             \\
Cúbico                                                  & 23                             & 432                             \\ \hline
\end{tabular}
\caption{Grupos de punto enantomórfico en cada sistema cristalino en la notación de Hermann-Mauguin \cite{Hermann2015notation}.}
\label{Tab. quiral}
\end{table}